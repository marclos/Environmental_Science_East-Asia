\chapter{Ores and Energy in the Lithosphere}\label{ch:ores}
\chapterauthor{Eliana Goehring and Marc Los Huertos\footnote{Statement of Contributions: Eliana Goehring wrote the sections on the geologic formations and methods of extraction of coal, oil and gas. Alison Hong started a provocative description of the issues with gold mining in the Phillipines. Later, Los Huertos decided to put these mining activities in a geologic context and revised the fossil fuel chapter that links to development of a population dependent on high density of energy.}}

\section{Energy: Industry, Food, and Population}

\subsection{Malthus and New Limits}

In the XXXX, Malthus argued that human population growth was exponential while resource growth was linear --- and a point had been reached where human populations would outstrip the capacity to get their basic needs met and misery would (or already had) fallen on the poor who would live destitute. He had some rather unsavory solutions to this conclusion --- stop feeding the poor with charity because they will just have more babies that would produce even more misery. 

In some ways, Malthus' observations remain and undercurrent in environmental views --- that humans will outstrip the resources of the Earth and even undermine the Earth's capacity to support non-human populations or even more poignant to many our own population.

\newglossaryentry{fossil fuels}{
	name=Fossil fuels, 
	description={are hydrocarbons, meaning they are composed of hydrogen and carbon. This term typically includes coal, crude oil, and natural gas which all started to form millions of years ago from plant and animal remains.}
}

However, in the late 1800s the predictions that humans would face a shortage of fire wood stimulated the US to take an active role in forest management to maintain a sustainable supply. However, with the use of coal and other fossil fuels humans were released from a carrying capacity on the Earth. In a strange way, then as discussed in the previous chapter, it is not the short supply of \gls{fossil fuels} that might limit the development of human populations, but the waste product of this resource in the form of CO2 that will undermine our success as a species. 

\begin{figure}[h]
	\centering
		\includegraphics[width=1.00\textwidth]{graphics/energy_population.jpg}
	\caption{energy and population, demonstrating the changes -- although we need to be careful to not make assumptions to about causation --- correlation does not necessarily mean causation!}
	\label{fig:energy_population}
\end{figure}

In terms of the impact on the world, we have...the ``density of power'' demonstrates the value of fossil fuels (Figure~/ref{fig:Power-Densities}), in spite of their areal needs, they remain a compelling source of energy. So, let's talk about fossil fuels and their geologic formations, mining, and processing.


\begin{figure}[htbp]
	\centering
		\includegraphics[width=1.00\textwidth]{graphics/Power-Densities.jpg}
	\caption{2222}
	\label{fig:Power-Densities}
\end{figure}


\section{Fossil Fuels: Coal, Natural Gas, and Oil}

\subsection{Fossil Fuels and the Environment}

Typically when fossil fuels are studied within the realm of environmental science, we focus on how relying on them for energy contributes to greenhouse gases. In this section, we are going to delve into other ways in which fossil fuels affect the environment. By focusing on the extraction processes of different types of fuels, we will discover the ways in which surrounding ecosystems are directly affected by mining through land subsidence and ground water pollution. We will focus on China, as this is the largest fossil fuel producer and user in all of Asia. 

For example, coal deposits throughout China are burning underground, releasing huge amounts of greenhouse gases, fundamentally changing landscapes, and having serious economic impacts. An estimated 15--20 million tons of coal are burning annually in northern China through these inadvertent fires \citep{kuenzer2007uncontrolled}.

% Citeint ``China: World? Largest Energy Consumer and Greenhouse Gas Emitter"
%   \1 China 2012, coal is 66\% of energy, fossil fuels over 90\%
%   \1 China is estimated to contain over 90 billion tonnes of coal reserves. In 2016, the country produced 2.62 billion tonnes of coal. CITE(world energy council)

%%%%%%%%%%%%%%%%%%%%%%%%%%%%%
%What are fossil fuels
%%%%%%%%%%%%%%%%%%%%%%%%%%%%%

\section{Fossil Fuels as Geologic Processes}

\subsection{What are fossil fuels?}

\Gls{fossil fuels} are hydrocarbons, meaning they are composed of hydrogen and carbon. This term typically includes coal, crude oil, and natural gas which all started to form millions of years ago from plant and animal remains. A combination of extremely high temperatures, high pressures, and millions of years allowed these organisms to be transformed into fossil fuels. 

\subsection{Coal Formations}

The first trees evolved around 360 million years ago at the beginning of the Carboniferous period. These ancient trees are the basis for our coal. Coal formation begins with thousands of years of plant accumulation. This process starts with plants in wetland environments dying and beginning to decompose. 
This plant matter is then buried beneath new layers of plants which later die and add to the layers of organic matter beneath them. The resulting partially decomposed plant matter is known as \gls{peat}. In tropical climates, the rate of peat accumulation is estimated to be about 2 meters every century.

Fossil fuel formations are were created in sedimentary basins. For example, the most significant coal formations originated from the detritus of trees, ferns, and other plants that were growing during the Carboniferous ``coal-bearing'' Period 290--360 million years ago. A combination of environmental factors, such as the evolution of large woody trees, allowed for coal deposits to begin to form during this time period. Lesser formations of coal continued through the Permian and Mesozoic Eras, 290--250 and 250--65 million years ago, respectively. Coal deposits that are less than 65 million years in age typically yield low quality coal, as they have not have enough time to fully transform into high-grade coal. 

\begin{figure}[htb]
	\centering
		\includegraphics[width=1.00\textwidth]{graphics/CoalFormation.png}
	\caption{Coal formation processes: imagine a nifty graphic probably with a flow-chart vibe, showing the different levels in the coal formation process and the corresponding increasing carbon concentrations}
	\label{fig:CoalFormation}
\end{figure}

As peat transforms to coal (Figure~\ref{fig:CoalFormation}), it is further compacted to be about one-tenth of its original depth. As peat is increasingly buried, the resulting pressure transforms it into \textbf{lignite} which is low quality coal. When comparing the appearance of peat and lignite, we would find peat to contain some undecomposed plant material which lignite lacks. Additionally, lignite forms distinct geological layers \citep{xie2015geological}.

The pressure and temperature continue to increase, allowing the low quality coal to progress to an intermediary sub-bituminous coal and then \textbf{bituminous coal}, the most widely spread type of coal.

Finally, with increasing pressure and a few more million years the high-grade coal, known as \textbf{anthracite}, is formed. The different types of coal contain varying concentrations of carbon: anthracite has the highest concentration which ranges from 86--98\% carbon, while lignite contains about 65--70\% carbon \citep{xie2015geological}.

\subsection{Petroleum and Natural Gas Formation}

In a comparable manner to coal, petroleum and natural gas are formed over long periods of time as organic materials are compressed and heated. We previously saw that coal is formed from decaying terrestrial plants; in contrast, petroleum and natural gas originate primarily from marine plankton.

\begin{figure}[ht]
\centering
    \includegraphics[width = 1\textwidth]{petroform.png}
    \caption{Formation of oil and gas reservoirs. Source: The Norwegian Petroleum Directorate }
    \label{fig:petroform}
\end{figure}

As the phytoplankton died, they sank to the seafloor and accumulated in the oxygen-free environment as sediment deposited on top of them. Similar to the decay process in wetlands, we often speak of diagensis in the marine systems that begin the breakdown process of organic matter on the marine floor.Over time, they became buried over time and like coal were transformed by heat and pressure. 

\section{Extracting Fossil Fuels: Coal Mining}

The first step in coal mining is, of course, is to find the stuff!  Some countries have massive reserves, e.g. the US and China have the most extensive reserves. 

Coal reserves are found throughout China, with recoverable reserves estimated at 324.1 billion tons. Figure~\ref{fig:coallocation} shows where these coal reserves are located throughout the country. The vast majority of this coal is not accessible via surface mining; in fact, only about 5--7 percent can be secured with this method. As of 2010, China was producing coal at a rate of 2.24 billion tons per year, of which surface mining accounted for about 9\%. 

\begin{figure}[ht]
\centering
    \includegraphics[width = 0.70\textwidth]{coallocation.png}
    \caption{Map of China's technically recoverable coal reserves by province. }
    \label{fig:coallocation}
\end{figure}

\subsubsection{Surface Coal Mining}

There are three distinct surface mining methods that are presently employed in China: open pit mining, area type mining, and contour mining. 

Openpit mining... 


\begin{figure}[ht]
\centering
    \includegraphics[width = 1.0\textwidth]{openpit.jpg}
    \caption{An open-pit coal mine, located in Fushun, in northeast China's Liaoning province. Image from Yan Bo of Zuma Press. }
    \label{fig:openpit}
\end{figure}

Area type mining is commonly employed in surface coal mines established after 1980 in China. 

In China, area type mining is characterized by the removal of coal seams ranging in thickness from 6 to 30 meters, with some being over 100 meters in thickness. 

Contour mining...



\citep{ji2012surface}


\subsubsection{Underground Coal Mining}


\subsection{Environmental Effects of Coal Mining}

\subsubsection{Fires in China}

Underground coal fires are not a new phenomena. Lightning strikes, grass and forest fires, and spontaneous combustion have all been contributing to coal fires for the past few million years \citep{ceycoal}. It is the frequency of these fires that has significantly increased due to human activity over the past century as more coal deposits are exposed for mining. 

As coal deposits are consumed by fires below ground, \textbf{subsidence} often becomes apparent above ground. Land subsidence refers to the gradual settling or sudden sinking of the Earth's surface. 

%%Cite USGS Groundwater Information: Land Subsidence
The landscape changes in areas surrounding subsurface coal fires are not subtle. Cracks induced by subsidence can measure up to a few meters wide, hundreds of meters deep, and can continue onward for several kilometers in length \citep{stracher2007coal}. %\citep{stracher2004coal}.

These cracks, which often occur on a slightly smaller scale, are part of a positive feedback loop (See Section~\ref{ssub:feedback}). %%%Cite Colin's section about feedback%%%
As coal deposits burn below ground, they create cracks in the ground above them. This gives the fire greater access to oxygen thereby promoting combustion. Consequently the coal continues to burn, causes more subsidence, more burning, and so on. Figure~\ref{fig:undergroundfire} shows how the underground coal fires cause the ground above them to subside and greater air flow increases the fire's capabilities. 


\begin{figure}[ht]
\centering
    \includegraphics[width = 0.70\textwidth]{undergroundfire.png}
    \caption{Underground coal fire. Schematic showing subsidence and pollution that occurs with underground coal fires (Source: TBD). }
    \label{fig:undergroundfire}
\end{figure}

Coal mining does simply occur in uninhabited areas. The effects of subsidence are apparent in Da Antou, located in Shanxi province about 650 kilometers southwest of Beijing. This small village sits atop a mountain that is littered with underground coal mines. In 2007 the majority of the 200 houses in Da Antou were cracked, and more than a dozen have been declared unfit to be inhabited since 2005. Chen Xiao'e, a former resident of Da Antou, describes how her windows started to shatter before cracks appeared, several centimeters in width, in the walls of her three-year-old brick home. Finally the floor buckled and the house became entirely unsafe to live in. 

\section{Oil and Natural Gas Drilling}

Drilling...


\section{Predictions of Peak Oil}


\subsection{Hydraulic fracturing (fracking)} 

Hydraulic fracturing, known more commonly as \textbf{fracking}, is an extraction method used to remove gas and oil from within impermeable shale rock. 

This method requires vertically drilling downward for about 2 km before turning to drill horizontally for as far as 3 km.  The fracking process itself is based in expelling fracking fluid, known as \textbf{slickwater}, at high pressures through small perforations in the horizontal pipe. The slickwater, composed of water, sand, and an assortment of additives, is forced out of the horizontal pipe at pressures above 600 atm creating microfractures for up to 50 m in the surrounding rock.

There are a number of environmental concerns surrounding fracking. 
\begin{itemize}
\item Water usage 

The fracking process requires extremely large volumes of water. INSERT NUMBER. Sourcing this much water can put a stress on surrounding surface and ground water reserves, particularly in desert regions such as INSERT LOCATION.
\item Water contamination 

The flowback liquid from fracking contains the original additives and may also contain heavy metals, radioactive material, and other toxins. Poorly constructed wells or other means of spilling this liquid could contaminate surrounding surface and ground water sources. 

\item Subsidence

As with other forms of fossil fuel extraction, removing and altering material from below ground tends to cause subsidence.

\item Methane

\end{itemize}


\section{A Fossil Fuel Addition}

\emph{The goal of this section is to show that while there are serious environmental/health effects associated with fossil fuels, we can't simply stop using them. I'll show that China, while incredibly dependent on coal, is actually one of the most efficient countries at processing coal. }

Let's now take a step back and briefly examine the economic and political reasons why countries like China are so dependent on fossil fuels and in particular, coal.  

While China is incredibly dependent on coal, its coal-fired power plants are significantly more efficient that those in the United States. In this context, efficiency refers to the amount of coal consumed per unit of power produced, and is thus related to gas emissions. 

Rapid growth, etc, etc, etc, relatively cheap option, reserves, etc

\subsection{Fossil Fuels and Asia}

\subsection{Fossil fuel production by country}

\subsection{Does Development Depend on Burning Fossil Fuels?}


%%%%%%%%%%%%%%%%%%%%%%%%%%%%%
%The Economics, my main focus
%%%%%%%%%%%%%%%%%%%%%%%%%%%%%




% \section{Underground Coal Fires in China} \cite{stracher2004coal}
%   \1 Cracks induced by subsidence measuring up to ``several kilometers long, tens of meters wide, and hundred of meters deep"
%       \2 Positive feedback loop- cracks increase oxygen circulation thereby promoting combustion 
%   \1 Economic loss estimated to be as high as \$25-250 million USD (Prakarsh)


\section{Mining Metals}

\subsection{Aluminum, Copper, and Rare Earth Metals}


\subsection{Gold in the Philippines}

The Philippines had the second largest gold reserve in world and employs approximately 200,000-300,000 people, most of who work in small scale mining operations, making approximately \$5/day (Pri)[??]

Many family miners, independent miners who make up the 70-80\% from small mining operations in the country. 


Large scale: \$70 billion 2014 (HWR)

90\% smuggled illegally

2015: Gov simplified process of gaining license 

But the Phillipines on just one region that has been subjected to the demand of gold and the re-occuring environmental and human tolls from the economic demand for gold. It drove some of the worst human behaviors from massacres, poisoning and destruction of whole groups of people via slavery and colonization. Although one could argue that the conditions today in the Phillipines is much better that the mines in the Andes or Brazil in the 1800s, but we would still be confronted with a terribly injustice system. To make a difference, we need to better appreciate why are these ores found unevenly and what are the processes to extract them and finally, how are the refined products used. 


\subsection{Geology and Gold}

\subsubsection{Mining and Processing Gold}

\subsection{Mining Methods and Ecological Impacts}

\subsubsection{Large Scale Processing}

\subsubsection{Small Scale Operations}

\citet{drasch2001mt} \ldots

Working small scale in families... (Figure \ref{fig:famliymining}).

\begin{figure}[h]
\includegraphics[width=\textwidth]{Family_Mining_Phillipines_SourceUnknwn}
\label{fig:famliymining}
\caption{Insert Caption Here: \ldots (With Permission ??).}
\end{figure}

\citep{de2016copper}

De la Torre, JB, Claveria RJ, Perez RE, Perez TR, Doronilla Al. 2016. Copper uptake by Pteris melanocaulon Fee from a Copper-Gold mine in Surigao del Norte Philippines. CRC Press LLC. 

Saludes, Mark. September 29, 2015. Hazardous Child Labor in Small-Scale Gold Mining in the Philippines. Human Rights Watch; [September 29, 2015; February 6, 2018].

\citet{santos1974mineral}

Santos G. 1974. Mineral Distribution and Geological Features of the Philippines. In: Petrascheck W.E. (eds) Metallogenetische und Geochemische Provinzen / Metallogenetic and Geochemical Provinces. �sterreichische Akademie der Wissenschaften Schriftenreihe der Erdwissenschaftlichen Kommission, vol 1. Springer, Vienna

Wernick, Adam. June 3, 2014. In the Philippines, underwater gold mining comes with small payoffs and big risk. PRI; [June 3, 2014; February 6, 2018]. 

\citet{drasch2001mt}

\section{Asia's Appetite for Sand}

Larson, Christina. "Asia's hunger for sand takes toll on ecology." (2018): 964-965.


%\bibliography{../references}



