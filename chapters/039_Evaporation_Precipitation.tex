\chapter{Heat Transfers: Evaporation, Transpiration, and Precipitation}\label{ch:evapotranspiration}
\chapterauthor{Marc Los Huertos}

\section{Hydrological Cycle}

\subsection{Solar Radiation as a Driver the Hydrologic Cycle}

NEED TO MAKE SURE THIS ISN'T BORING!

As describe in Chapter~\ref{}, the sun heats the planet's atmosphere. While keeping our planet well above 0 degrees C made life as we know it possible, it does alot more. 

First, the heat drives weather patterns, and for our purposes now, it drives the hyrologic cycle.  

\subsection{Heat and Evaporation}

\subsubsection{Evaporation and Evapotranspiration}

\subsubsection{Potential Evaporation}

\subsection{Precipitation: Rain and Snow}

\subsubsection{Latitude and Rain}

\subsubsection{Orographic Effects}

\subsubsection{Desert Regions}

\section{Climates of Asia}

\subsection{Temperate Zones of Asia}

Much of east Asia is temperate --- but as one moves south the climate become increasingly tropical. The temperate zone is defined by ... 

The vegetation is adapted to XXX 

\subsection{Tropical Zones}

The tropical zone, loosely defined as the region between the Tropic of Cancer and Tropic of Capricorn (latitude XX and -XX). The region has rainfall that often exceeds XX mm per year and a mean annual temp of XX and at lower elevations never any frost events. 

\section{Static versus Dynamic Views of the Climate System}


\section{Conclusion}

The upward and downward transfer of heat and rain define the basic weather patterns on planet Earth, but as we'll see in Chapter~\ref{ch:climate} the lateral transfer of heat and moisture with winds provide yet another process that help use understand how weater and climate works. Once we understand these additional factors, then we can better appreciate the complexity of climate and its dynamic nature.  Stay tuned.