\chapter*{Preface}

\section{Guiding Principles}

In this project, the EnviroLab Asia fellows have written a textbook that highlights examples of environmental processes that occur along a vertical axis. 

Each student contributes to one theme, composed of two examples that highlight environmental issues of East Asia. However, our goal is not to blame East Asian, but to point to linkages of how a range of globalized economy contribute to these environmental problems. 

In the end, it would be useful for us to acknowledge we have some capacity to address these how these global linkages could be modified to reduce these environmental issues. 

\subsection{Goals}

Processes across horizontal boundaries define many environmental patterns that frame human interactions with the environment. How do humans impact processes that cross these boundaries and how do humans influence these ecosystem interface?

\subsection{Rationale}

\subsection{Activity}

Each group will be composed of two students, that will become experts and teach their classmates on the topic. 

\section{East Asia and the World}







\section{Acknowledgments}

Everyone in the world!





\section{Thoughts for the Contributors}

\subsection{Why Learn \LaTeX?}

\subsection{How to Learn \LaTeX?}

This template will NOT teach you how to use \LaTeX! To accomplish that, we'll rely on some great online resource that you can find on the project page. 

Instead this section of the document is designed to demonstrate how our textbook will look, feel, and ultimately how we contribute to the project.

This document also compiles all of our projects into a single PDF, where each chapter is composed of a input tex file.

\subsection{Noting Your Contribution}

Because this is an ongoing project, you should record your contribution to each chapter -- but also let go of these contributions at some point; Others might revise and their authorship might take some precedence, so you should both invest in the product but also be willing to detach from the final outcome as others contribute. This will feel uncomfortable at times, but please note from the beginning this is a social process and as such subject to negotiation. Please be generous to the authors that laid the foundation and be respectful of those that follow. 

\section{Setting Up Book Project--Type Setting w/ \LaTeX}


\subsection{The Standard Latex Book Class}


\subsection{Structuring the Text with Nested Hierarchies}

You should divide up your chapter into sections and subsections...Use the \verb"\section{Section}" command for major sections, and the \verb"\subsection{Subsection}" command for subsections, etc.

We may dispense with subsubsections, but for now they might be useful. We will use section, subsection, and subsubsection to break up the topic into bite sizes. 

NOTE: if you have nested levels, they MUST be followed by the lowest level in the section before a paragraph is started -- in contrast to what is shown above!

\subsection{Font Changes}

SSelect a part of the text then click on the button Emphasize (H!), or Bold (Fs), or
Italic (Kt), or Slanted (Kt) to typeset \emph{Emphasize}, \textbf{Bold},
\textit{Italics}, \textsl{Slanted} texts.

You can also typeset \textrm{Roman}, \textsf{Sans Serif}, \textsc{Small Caps}, and
\texttt{Typewriter} texts.

You can also apply the special, mathematics only commands $\mathbb{BLACKBOARD}$
$\mathbb{BOLD}$, $\mathcal{CALLIGRAPHIC}$, and $\mathfrak{fraktur}$. Note that
blackboard bold and calligraphic are correct only when applied to uppercase letters A
through Z.

You can apply the size tags -- Format menu, Font size submenu -- {\tiny tiny},
{\scriptsize scriptsize}, {\footnotesize footnotesize}, {\small small}, {\normalsize
normalsize}, {\large large}, {\Large Large}, {\LARGE LARGE}, {\huge huge} and {\Huge
Huge}.

You can use the \verb"\begin{quote} etc. \end{quote}" environment for typesetting
short quotations. Select the text then click on Insert, Quotations, Short Quotations:

\begin{quote}
The buck stops here. \emph{Harry Truman}

Ask not what your country can do for you; ask what you can do for your
country. \emph{John F Kennedy}

I am not a crook. \emph{Richard Nixon}

I did not have sexual relations with that woman, Miss Lewinsky. \emph{Bill Clinton}
\end{quote}

The Quotation environment is used for quotations of more than one paragraph. Following
is the beginning of \emph{The Jungle Books} by Rudyard Kipling. (You should select
the text first then click on Insert, Quotations, Quotation):

\begin{quotation}
It was seven o'clock of a very warm evening in the Seeonee Hills when Father Wolf woke
up from his day's rest, scratched himself, yawned  and spread out his paws one after
the other to get rid of sleepy feeling in their tips. Mother Wolf lay with her big gray
nose dropped across her four tumbling, squealing cubs, and the moon shone into the
mouth of the cave where they all lived. ``\emph{Augrh}'' said Father Wolf, ``it is time
to hunt again.'' And he was going to spring down hill when a little shadow with a bushy
tail crossed the threshold and whined: ``Good luck go with you, O Chief of the Wolves;
and good luck and strong white teeth go with the noble children, that they may never
forget the hungry in this world.''

It was the jackal---Tabaqui the Dish-licker---and the wolves of India despise Tabaqui
because he runs about making mischief, and telling tales, and eating rags and pieces of
leather from the village rubbish-heaps. But they are afraid of him too, because
Tabaqui, more than any one else in the jungle, is apt to go mad, and then he forgets
that he was afraid of anyone, and runs through the forest biting everything in his way.
\end{quotation}

Use the Verbatim environment if you want \LaTeX\ to preserve spacing, perhaps when
including a fragment from a program such as:
\begin{verbatim}
#include <iostream>         // < > is used for standard libraries.
void main(void)             // ''main'' method always called first.
{
 cout << ''This is a message.'';
                            // Send to output stream.
}
\end{verbatim}
(After selecting the text click on Insert, Code Environments, Code.)


\subsection{Mathematics and Text}

\subsubsection{Warning: Special Characters}

When you use percent and ampersand symbols, hash tags, and other non-standard ASCII characters, \LaTeX will be very uncooperative. So, do yourself a favor and make sure you understand that these are used for special typesetting functions. To use them you have to ``escape'' and use commands to get them to do what you might usually expect!  \% \# \& \`e \~n `` and '' to show a few that do not reflect the key stroke you might expect. 

\LaTeX doesn't like a range of characters or they reserved for special behavior...

For example, the \# is used for tabs in a table environment. \% is used to make comments, thus stuff behind a \% is ignored. There are lots of others, but these come up the most.

\subsubsection{Creating equations}

One of the most powerful parts of \LaTeX is how it can be used to write complex equations, with all those symbols and Greek letters! This can be done inline $y = mx + b + \epsilon$ for fairly simple equations, or set apart for more complex equations:

\begin{equation}
\int_0^\infty e^{-x^2} dx=\frac{\sqrt{\pi}}{2}
\end{equation}

\subsubsection{Theorems, etc}
\begin{theorem}
(The Currant minimax principle.) Let $T$ be completely continuous selfadjoint operator
in a Hilbert space $H$. Let $n$ be an arbitrary integer and let $u_1,\ldots,u_{n-1}$ be
an arbitrary system of $n-1$ linearly independent elements of $H$. Denote
\begin{equation}
\max_{\substack{v\in H, v\neq
0\\(v,u_1)=0,\ldots,(v,u_n)=0}}\frac{(Tv,v)}{(v,v)}=m(u_1,\ldots, u_{n-1})
\label{eqn10}
\end{equation}
Then the $n$-th eigenvalue of $T$ is equal to the minimum of these maxima, when
minimizing over all linearly independent systems $u_1,\ldots u_{n-1}$ in $H$,
\begin{equation}
\mu_n = \min_{\substack{u_1,\ldots, u_{n-1}\in H}} m(u_1,\ldots, u_{n-1}) \label{eqn20}
\end{equation}
\end{theorem}
The above equations are automatically numbered as equation (\ref{eqn10}) and
(\ref{eqn20}).


\subsection{Lists Environments: Making bulletted, numbered, description lists}

We use special commands to create an itemized list.

You can create numbered, bulleted, and description lists
(Use the Itemization or Enumeration buttons, or click on the Insert menu
then chose an item from the Enumeration submenu):

\begin{enumerate}
\item List item 1

\item List item 2

\begin{enumerate}
\item A list item under a list item.

\item Just another list item under a list item.

\begin{enumerate}
\item Third level list item under a list item.

\begin{enumerate}
\item Fourth and final level of list items allowed.
\end{enumerate}
\end{enumerate}
\end{enumerate}
\end{enumerate}

\begin{itemize}
\item Bullet item 1

\item Bullet item 2

\begin{itemize}
\item Second level bullet item.

\begin{itemize}
\item Third level bullet item.

\begin{itemize}
\item Fourth (and final) level bullet item.
\end{itemize}
\end{itemize}
\end{itemize}
\end{itemize}

\begin{description}
\item[Description List] Each description list item has a term followed by the
description of that term.

\item[Bunyip] Mythical beast of Australian Aboriginal legends.
\end{description}

\subsection{Theorem-Like Environments}

The following theorem-like environments (in alphabetical order) are available
in this style.

%\begin{acknowledgement}
%This is an acknowledgement
%\end{acknowledgement}

%\begin{algorithm}
%This is an algorithm
%\end{algorithm}

%\begin{axiom}
%This is an axiom
%\end{axiom}

%\begin{case}
%This is a case
%\end{case}

%\begin{claim}
%This is a claim
%\end{claim}

%\begin{conclusion}
%This is a conclusion
%\end{conclusion}

%\begin{condition}
%This is a condition
%\end{condition}

%\begin{conjecture}
%This is a conjecture
%\end{conjecture}

%\begin{corollary}
%This is a corollary
%\end{corollary}

%\begin{criterion}
%This is a criterion
%\end{criterion}





\begin{example}
This is an example
\end{example}

\begin{exercise}
This is an exercise
\end{exercise}

%\begin{lemma}
%This is a lemma
%\end{lemma}

%\begin{proof}
%This is the proof of the lemma.
%\end{proof}

%\begin{notation}
%This is notation
%\end{notation}

%\begin{problem}
%This is a problem
%\end{problem}

%\begin{proposition}
%This is a proposition
%\end{proposition}

%\begin{remark}
%This is a remark
%\end{remark}

%\begin{summary}
%This is a summary
%\end{summary}

%\begin{theorem}
%This is a theorem
%\end{theorem}

%\begin{proof}
%[Proof of the Main Theorem]This is the proof.
%\end{proof}

%\subsubsection{``Child'' Rnw Contributions}

%This is a chapter that we can input into the text... you will each create a chapter without the preamble and begin and end document... that can be integrated into a single book! 

\subsection{Peer Review Commenting}

You can put your comments in square brackets and in color for things that need help. \textcolor{red}{[This section is confusing, I am not sure what commenting means.]}

\subsection{Adding Figures, etc}

\subsection{Using Rnw Files -- Deprecated}

Originally, I used R and Rnw files that were converted to tex files, in Rstudio. However, this step was too complicated for students, so I now recommend the use of tex files and use Rstudio to create figure files as a separate process. Thus, I will no longer explain the process in this document. 

\subsection{Creating a floating figure}

This is my floating figure (Figure \ref{fig:plot}).

\begin{figure}
\label{fig:plot}
\caption{My plot's caption is here!}
\end{figure}

\subsection{Creating References, Indicies, and Glossaries}

\subsubsection{Bibliography generation}

This document was produced in RStudio using the knitr package \citep{knitr2013} by \url{http://texblog.org}. 

Also try \citet{fenner2018} to create an author title. 

Currently, we are using the ecology.bst, but it has trouble with misc type of references, so I will changing this in 2019. 

\subsubsection{Creating glossary words}
 
\newglossaryentry{peat}{
	name=Peat, 
	description={peat is cool.}
}

\begin{definition}
This is a definition and the word is use in an glossary, e.g. \gls{peat}. \Gls{peat} is when you want to capitalize the defined word without having to re-define a capitalized version, the only downside of case sensitivity in \LaTeX.
\end{definition}