\chapter[Tropospheric Heat Transfers]{Tropospheric Heat Transfers: Evaporation, Transpiration, and Precipitation}\label{ch:evapotranspiration}
\chapterauthor{Marc Los Huertos}

\section{Hydrological Cycle}

\subsection{Solar Radiation as a Driver the Hydrological Cycle}

The suns energy powers the hydrologic cycle in Earth. And the hydrology effectively creates weather. 

As describe in Chapter~\ref{ch:earth_sun}, the sun heats the planet's atmosphere. While keeping our planet well above 0~\degree C made life as we know it possible, it does alot more. 

First, the heat drives weather patterns, and for our purposes now, it drives the hyrologic cycle.  

\subsection{Heat and Evaporation}

\subsubsection{Evaporation and Evapotranspiration}

The heated atmosphere allows the air to hold more air -- thus increases the rate of evaporation. The water vapor in the atmosphere can then be redistributed by air currents.

Water is lost directly from plants as well -- but it forms a tight linkage to soils, where evapotranspiration is the loss of water via stomata in leaves that has traveled through the vascular system of plants, the roots and soil. 

\subsubsection{Potential Evaporation}

However, just because the air is warm enough and it can 'hold' more water doesn't mean it will. The difference between water that is actually evaporated and what could be evaporated is termed as potential evaporation. When biomass is related to potential ET we see a strong correlation. 


\subsection{Precipitation: Rain and Snow}

\subsubsection{Temperature}

Water the atmospheric temperature falls below a certain temperature... 
thus, the atmosphere can no longer hold water. 

\subsubsection{Precipitation and Elevation}

These changes in temperature can depend on the height of the air mass. As air masses heat they also rise, as the air mass go up in temperature, the temperature declines and can'tholdwater, thus, it rains.


\subsubsection{Orographic Effects}

\newglossaryentry{orographic effects}{
	name=orographic effects, 
	description={is cool.}
}

\Gls{orographic effects}...

\subsubsection{Rain and Latitude}


\subsection{Tropical Zones}

The tropical zone, loosely defined as the region between the Tropic of Cancer and Tropic of Capricorn (latitude XX and -XX). The region has rainfall that often exceeds XX mm per year and a mean annual temp of XX and at lower elevations never any frost events. 

The vegetation is adapted to XXX 

\subsubsection{Desert Regions}

\subsection{Temperate Zones}

\subsection{Monsoons}

Much of east Asia is temperate --- but as one moves south the climate become increasingly tropical. The temperate zone is defined by ... however, we are now confronted with the monsoon -- which was coined for the weather of east asia, we will explore the monsoon weather as we begin to unpack weather patterns with more detail (Chapter~\ref{ch:climate}). 


\section{Static versus Dynamic Views of the Climate System}

\subsection{Paleoclimates versus Climate Change}

As shown in Chapter \ref{ch:earth_sun}, the Earth has experienced dramatic shifts in climate over a record of 100 thousand years. Thus, the climate of the Earth, the average of weather is non-stationary or static. The Earth's climate is dynamic and sensitive to various factors (forcings) that lead to warming or cooling.

Since the Holocene (~18,000 years ago), the Earth has been warming. The warming has driven changes in weather patterns, as the patterns of how heat is distributed can be disrupted. The shift to a warming climate became with metaling glaciers and caused by XXX???, but even this warming has been interupted, by various events, such as the ``Little Ice Age''. What cause this event is subject to much debate, but was likely caused by...

\subsection{Climate in the Anthropocene}
 
Meanwhile, the Earth's climate has been on an accerated increase in temperature, over and above the warming post the Holocene. Although the exact amount that is due to natureal cycle and anthropogenic causes might be considered controversial by some, we have a pretty good idea that the average energy entering the Earth is higher because of greenhouse gases. This increases the amount of energy in our earth's weather systems. We will dicuss the implications of these changes in Chapter \ref{ch:climate}. 

\section{Conclusion}

The upward and downward transfer of heat and rain define the basic weather patterns on planet Earth, but as we'll see in Chapter~\ref{ch:climate} the lateral transfer of heat and moisture with winds provide yet another process that help use understand how weater and climate works. Once we understand these additional factors, then we can better appreciate the complexity of climate and its dynamic nature.  Stay tuned.