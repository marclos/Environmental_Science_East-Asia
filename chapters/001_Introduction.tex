\chapter{Introduction}

\section{Goals}

\subsection{Guiding Principles}

The introduction is entered using the usual chapter command. Since
the introduction chapter appears before the \verb|mainmatter| TeX
field, it is again an unnumbered chapter. The primary difference
between the preface and the introduction in this sample document
is that the introduction will appear in the table of contents and
the page headings for the introduction are automatically handled
without the need for the \verb|markboth| TeX field. You may use
either or both methods to create chapters at the beginning of your
document. You may also delete these preliminary chapters.

In this project, the EnviroLab Asia fellows have written a textbook that highlights examples of environmental processes that occur along a vertical axis. 

Each student contributes to one theme, composed of two examples that highlight environmental issues of East Asia. However, our goal is not to blame East Asian, but to point to linkages of how a range of globalized economy contribute to these environmental problems. 

In the end, it would be useful for us to acknowledge we have some capacity to address these how these global linkages could be modified to reduce these environmental issues. 

\subsection{Goals}

Processes across horizontal boundaries define many environmental patterns that frame human interactions with the environment. How do humans impact processes that cross these boundaries and how do humans influence these ecosystem interface?

Non-declintionist views...but also no technological optimism. 

\subsection{Rationale}


\section{Acknowledgments}

\subsection{EnviroLab Asia}

We begin with the EnviroLab Asia scholars at the Claremont Colleges and thank them for their contributions over the years. In addition to our East Asian partners, in particular Yale NUS and the work of Brian McAdoo. 

\subsection{Faculty}

Everyone in the world!






\subsection{Activity}

Each group will be composed of two students, that will become experts and teach their classmates on the topic. 


\section{Patterns and Processes in Scale}

\subsection{Project Themes}

How do environmental interfaces in East Asia influence human infrastructure and vice versa?

\begin{description}
	\item[A. Space-Atmosphere Interface] These might include the Earth's energy budget, drivers of climate and climate change, UV radiation and ozone processes.
	
	\item[B. Mantle-Lithosphere] The Earth's core and mantel are active agents in the definition and processes that influence tectonic and landforms -- but also frame how, where, and when natural disasters occur, but they also tell use much about the resources, fuels and ores, used in the development of modern industrialization.  
	
	\item[X. Hydrologic Cycle]	
	
	\item[D. Soil-Water-Atmosphere Interface: Plant Biomass and Diversity] Light and gas exchange are the most common discussed fluxes across the air-water interface, but various dead and living biotic material also comes from or is deposited into the water column. While rainfall drives the productivity of terrestiral plant growth, the diversity is thought to depend on a range of processes.

	\item[C. Soil-Air Interface] Pedogenesis of soils includes climate (rain, air temperatures), atmospheric deposition of dust and other particles and gas exchange. 
	
	\item[X Carbon in Air, Plants, and Soil: Peatlands]
	
	\item[E. Benthos-Water Interface] The flow of ions, organic matter deposited, gas and nutrient exchanges.

	\item[X. Isostasy and Continental Plates]
	
	\item[X. Tectonic Hazards]
	
	\item[X. Weather Patterns and the Dynamic Climate System]
	
	\item[X. Hydrologic Hazards: Floods and Droughts]
	
\end{description}	

\section{East Asia and the World}