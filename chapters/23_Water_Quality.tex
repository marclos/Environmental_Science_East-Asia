\chapter{Water Quality}\label{ch:water_quality}

\chapterauthor{TBD\footnote{Statement of Contributions: Betel and Luyi wrote a section about Arsenic for another chapter, but we moved it to this chapter to focus on water quality as a whole...}}

\section{Arsenic and Water Borne Disease}



\section{Arsenic}

\subsection{Arsenic in the Mekong Aquifers}

How are the geographical features of the Mekong Delta providing resources to humans and furthermore, what elements of their formation pose challenges to how Vietnamese people will interact with these features in the future?

Erban et al. (2013)\footnote{fix citation} investigate the challenges being posed by the overexploitation of aquifers in South and Southeast Asia, with a specific focus on Vietnam. Their research focus is a large (>1,000 km2)  area of the Mekong Delta in Vietnam. Arsenic in groundwater continues to threaten human health in this region. It is a colorless, odorless, and tasteless poison that is toxic to humans if they are exposed to it over a long period of time (Ravenscroft, year)\footnote{fix citation}. The poison occurs most commonly in sands deposited by large rivers, with the worst cases recorded in the tropical basins of Asia. Arsenic-contaminated groundwater is found in the unconsolidated sediments and sedimentary, igneous and metamorphic rock which age in range from thousands of years to billions. In most cases, high levels of arsenic in the groundwater also means there is arsenic in the soils. The effects of arsenic in groundwater is both additive and cumulative because plants can absorb arsenic through the roots. The subsistence rice economies of Asia, whereby rice is irrigated with arsenic-contaminated water, present the worst cases. The deep aquifers are presumed to be sources of pathogen- and arsenic-free water. 

\subsubsection{Historical Context}

Pliocene-Miocene-age aquifers in the Mekong Delta are contaminated at depths of 200-500 m and they have been heavily exploited. Generally, seven main aquifers are heavily pumped in the Mekong Delta, varying from the Pliocene to the Miocene age. The Division for Geological Mapping for the South of Vietnam, who used a combination of scientific techniques  (i.e. mud logging of drill cuttings), determined the delineation of aquifers and their ages, used by Erban et al.\footnote{fix citation} in this work. Pumping wells is a process that existed in Vietnam since the 1900s, but became more common after the 1980s. The dissolved concentrations of arsenic are highest near the surface, closer to the Mekong river and its tributaries, reducing greatly with distance. Extreme arsenic concentrations are found in numerous hot spot regions. Erban et al. found that deep wells located in the focus areas of the research in the Mekong area showed more contamination than in other parts of Southeast Asia.
  
An analysis of satellite-based radar images recorded from 2007 to 2010 shows how intensive groundwater extraction has caused land subsidence of up to 3cm/year. Furthermore, transient 3D aquifer simulations show similar subsidence rates with a total subsidence of up to 27 cm since 1988.  Erban et al. (2013)\footnote{fix citation} propose a causal mechanism that explains the long-term effect of groundwater extraction on interbedded clays. The groundwater extraction causes these interbedded clays to compact and expel water containing dissolved arsenic or arsenic-mobilizing solutes, examples being dissolved organic carbon and competing ions, into deep aquifers over decades.
  
The subsequent threat is that deep, untreated groundwater in the Mekong Delta Region will no longer be a safe source of drinking water. The pumping-induced clay compaction was measured as land subsidence. Erban et al. present a conceptual model that explains the vertical distribution of arsenic in the Mekong Delta according to how it developed historically. Starting in the Miocene age and tracking to the present, the researchers explain how fresh clays rich in arsenic and organic compound were deposited widely. The slow diffusion out of contaminated pores and dissolution of the solid-phase arsenic supply allowed for a continuous deep arsenic load in deep clays. Recently, the overpumping of low-arsenic, deep aquifers resulted in causing water carrying arsenic-mobilizing solutes to be squeezed out of the deadflow storage of clays to neighbouring aquifers. 


Kintisch, Eli. "Department of State's air pollution sensors go global." (2018): 248-249.